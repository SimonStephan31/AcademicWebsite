% Options for packages loaded elsewhere
\PassOptionsToPackage{unicode}{hyperref}
\PassOptionsToPackage{hyphens}{url}
%
\documentclass[
]{article}
\usepackage{amsmath,amssymb}
\usepackage{iftex}
\ifPDFTeX
  \usepackage[T1]{fontenc}
  \usepackage[utf8]{inputenc}
  \usepackage{textcomp} % provide euro and other symbols
\else % if luatex or xetex
  \usepackage{unicode-math} % this also loads fontspec
  \defaultfontfeatures{Scale=MatchLowercase}
  \defaultfontfeatures[\rmfamily]{Ligatures=TeX,Scale=1}
\fi
\usepackage{lmodern}
\ifPDFTeX\else
  % xetex/luatex font selection
\fi
% Use upquote if available, for straight quotes in verbatim environments
\IfFileExists{upquote.sty}{\usepackage{upquote}}{}
\IfFileExists{microtype.sty}{% use microtype if available
  \usepackage[]{microtype}
  \UseMicrotypeSet[protrusion]{basicmath} % disable protrusion for tt fonts
}{}
\makeatletter
\@ifundefined{KOMAClassName}{% if non-KOMA class
  \IfFileExists{parskip.sty}{%
    \usepackage{parskip}
  }{% else
    \setlength{\parindent}{0pt}
    \setlength{\parskip}{6pt plus 2pt minus 1pt}}
}{% if KOMA class
  \KOMAoptions{parskip=half}}
\makeatother
\usepackage{xcolor}
\usepackage[margin=1in]{geometry}
\usepackage{graphicx}
\makeatletter
\def\maxwidth{\ifdim\Gin@nat@width>\linewidth\linewidth\else\Gin@nat@width\fi}
\def\maxheight{\ifdim\Gin@nat@height>\textheight\textheight\else\Gin@nat@height\fi}
\makeatother
% Scale images if necessary, so that they will not overflow the page
% margins by default, and it is still possible to overwrite the defaults
% using explicit options in \includegraphics[width, height, ...]{}
\setkeys{Gin}{width=\maxwidth,height=\maxheight,keepaspectratio}
% Set default figure placement to htbp
\makeatletter
\def\fps@figure{htbp}
\makeatother
\setlength{\emergencystretch}{3em} % prevent overfull lines
\providecommand{\tightlist}{%
  \setlength{\itemsep}{0pt}\setlength{\parskip}{0pt}}
\setcounter{secnumdepth}{-\maxdimen} % remove section numbering
\ifLuaTeX
  \usepackage{selnolig}  % disable illegal ligatures
\fi
\IfFileExists{bookmark.sty}{\usepackage{bookmark}}{\usepackage{hyperref}}
\IfFileExists{xurl.sty}{\usepackage{xurl}}{} % add URL line breaks if available
\urlstyle{same}
\hypersetup{
  pdftitle={Simon Stephan},
  hidelinks,
  pdfcreator={LaTeX via pandoc}}

\title{Simon Stephan}
\author{}
\date{\vspace{-2.5em}}

\begin{document}
\maketitle

\includegraphics{images/home/avatar.png}\\

Simon Stephan

Research Scientist in the field of Cognitive Science at the University
of Göttingen

\href{mailto:simon.stephan@psych.uni-goettingen.de}{}
\href{https://github.com/SimonStephan31}{}
\href{https://twitter.com/SimonStephan31}{}
\href{https://scholar.google.de/citations?hl=de\&user=_GsFvvAAAAAJ}{}

\hypertarget{about-me}{%
\section{About me}\label{about-me}}

I am a post-doctoral researcher at the University of Göttingen working
in
\textbf{\href{https://www.psych.uni-goettingen.de/de/cognition/team/waldmann}{Michael
Waldmann's}} Lab of
\textbf{\href{https://www.psych.uni-goettingen.de/en/cognition?set_language=en}{Cognitive
and Decision Sciences}}. I'm very much interested in how minds learn and
reason about the causal relationships in the world. Given that causal
relations are neither directly observable nor logically deducible, how
do reasoners manage to learn about causes and effect so successfully?
How do they use this knowledge to make predictions, diagnoses, or to
explain things? I'm also fascinated by the more general question of how
minds form and use categories.

On this website I share information about my academic life. You'll find
an up-to-date \textbf{\href{cv/simonstephan_cv.pdf}{CV}}, my
\textbf{\protect\hyperlink{publications}{publication list}}, and
information about recent/ upcoming presentations (e.g., conference
talks).

\hypertarget{interests}{%
\subsubsection{Interests}\label{interests}}

\begin{itemize}
\tightlist
\item
  Computational Cognitive Science
\item
  Philosophy of Science
\item
  Causal Cognition
\item
  Learning and Reasoning
\item
  Statistical Inference
\item
  Open Science
\end{itemize}

\hypertarget{education}{%
\subsubsection{Education}\label{education}}

\begin{itemize}
\tightlist
\item
  PhD in Psychology, 2019 University of Göttingen
\item
  MSc in Psychology, 2014 University of Göttingen
\item
  BSc in Psychology, 2012 University of Göttingen
\end{itemize}

\hypertarget{grants-honors-awards}{%
\subsubsection{Grants, Honors \& Awards}\label{grants-honors-awards}}

\begin{itemize}
\tightlist
\item
  2017 Computational Modeling Prize - Higher Level Cognition Awarded by
  the Cognitive Science Society
\item
  2016 Leibniz-ScienceCampus Grant, Project: The relationship between
  causal and moral judgments
\item
  2015 Leibniz-ScienceCampus Grant, Project: The role of intentions in
  children's and adult's causal ascriptions
\end{itemize}

\hypertarget{teaching}{%
\subsubsection{Teaching}\label{teaching}}

\begin{itemize}
\item
  Winter terms (2014/15 until 2021/22): Quantitative Methods I Seminar
  As part of the first year undergraduate psychology statistics class
\item
  Summer terms (2015 until 2022): Quantitative Methods II Seminar As
  part of the first year undergraduate psychology statistics class
\item
  Winter term 2022/23: Seminar on the principles of learning and
  behavior As part of the second year undergraduate psychology module
  ``Allgemeine Psychology II'' (General Psychology II)
\end{itemize}

The seminar \emph{Quantitative Methods I} covers basics of research
design and the application of hypothesis testing, data visualisation,
probability theory, descriptive and inferential data analysis, and power
analyses. \emph{Quantitative Methods II} focuses on the General Linear
Model and its applications (regression, ANOVA, contrast analyses,
multilevel models). Students learn to apply these methods with R and
RStudio.

An overview of the teaching resources (including teaching videos) is
given at: \url{https://quantigoettingen.github.io/quantigoettingen}

I also supervised a number of Bachelor and Master projects (see my
\href{https://www.simonstephan.com/cv/simonstephan_cv.pdf}{CV} for a
list).

\hfill\break

\hypertarget{selected-publications}{%
\section{Selected Publications}\label{selected-publications}}

\begin{itemize}
\item
  \textbf{Stephan, S.} (2023). Revisiting the narrow latent scope bias
  in explanatory reasoning. \emph{Cognition}, 241,
  105630.{\href{publications/Stephan_2023_RevisitingTheNarrowLatentScopeBias.pdf}{}}
  {\href{https://simonstephan31.github.io/revisit_nlsbias/index.html}{}}
  {\href{https://osf.io/n94ed/}{}}{\href{https://psyarxiv.com/a8cve/}{}}
\item
  \textbf{Stephan, S.}, Engelmann, N., \& Waldmann, M. R. (2023). The
  perceived dilution of causal strength. \emph{Cognitive Psychology},
  140,
  101540.{\href{publications/Stephan_Engelmann_Waldmann_2023_strengthDilution.pdf}{}}
  {\href{https://simonstephan31.github.io/The-Dilution-of-Causal-Strength/index.html}{}}
  {\href{https://osf.io/nrv7p/}{}}
\item
  \textbf{Stephan, S.}, \& Waldmann, M. R. (2022). The role of mechanism
  knowledge in singular causation judgments. \emph{Cognition}, 218,
  104924.
  {\href{publications/Stephan_Waldmann_2021_MechanismCognit.pdf}{}}
  {\href{https://osf.io/325pr/}{}}
  {\href{https://github.com/SimonStephan31/The-Role-of-Mechanism-Information-in-Singular-Causation}{}}
\item
  \textbf{Stephan, S.}, Tentori, K., Pighin, S., \& Waldmann, M. R.
  (2021). Interpolating causal mechanisms: The paradox of knowing more.
  \emph{Journal of Experimental Psychology: General}, 150(8), 1500-1527.
  {\href{https://doi.apa.org/fulltext/2021-13036-001.pdf}{}}
  {\href{https://osf.io/aqzps/}{}}
\item
  \textbf{Stephan, S.}, Mayrhofer, R., \& Waldmann, M. R. (2020). Time
  and singular causation - a computational model. \emph{Cognitive
  Science}, 44, e12871.
  {\href{https://onlinelibrary.wiley.com/doi/epdf/10.1111/cogs.12871}{}}
  {\href{https://osf.io/5yvs4/}{}} {\href{https://osf.io/n93bu}{}}
\end{itemize}

\begin{center}\rule{0.5\linewidth}{0.5pt}\end{center}

\hfill\break

\hypertarget{tutorials}{%
\section{Tutorials}\label{tutorials}}

I'm passionate about online experiments and created a small series of
YouTube-Tutorials where I show how to create a ``typical cognitive
science'' experiment using \href{https://www.jspsych.org/7.3/}{JsPsych}.

\begin{center}\rule{0.5\linewidth}{0.5pt}\end{center}

\hfill\break

\hypertarget{recent-talks-conference-presentations}{%
\section{Recent Talks \& Conference
Presentations}\label{recent-talks-conference-presentations}}

\hypertarget{section}{%
\subsection{2022}\label{section}}

\begin{itemize}
\item
  Flash talk presentation at
  \href{https://cognitivesciencesociety.org/cogsci-2022/}{CogSci 2022}:
  \textbf{The Perceived Dilution of Causal Strength} (July 2022, Toronto
  {[}remote{]}) {\href{publications/Dilution_Poster_2022.pdf}{}}
\item
  Poster presentation at \href{https://espp-spp-2022.com/}{SPP \& ESPP}:
  \textbf{The Perceived Dilution of Causal Strength} (July 2022, Milan,
  Italy) {\href{publications/Dilution_Poster_2022.pdf}{}}
\item
  Invited Talk {[}virtual{]} at the
  \href{https://www.ucl.ac.uk/pals/london-judgment-and-decision-making-seminars}{London
  Judgment and Decision Making Seminar}: \textbf{The interplay between
  covariation, temporal, and mechanism information in singular causation
  judgments} (Jan 2021, London)
\item
  Invited Talk {[}virtual{]} at the
  \href{https://www.ucl.ac.uk/pals/london-judgment-and-decision-making-seminars}{Computational
  Cognitive Science Group} at the University of Edinburgh: \textbf{The
  interplay between covariation, temporal, and mechanism information in
  singular causation judgments} (Jan 2021, London)
\end{itemize}

\hypertarget{section-1}{%
\subsection{2021}\label{section-1}}

\begin{itemize}
\item
  Poster presentation {[}virtual{]} at
  \href{https://cognitivesciencesociety.org/cogsci-2021/}{CogSci 2021}:
  \textbf{Evaluating general versus singular causal prevention} (July
  2021, Vienna)
  {\href{publications/Poster_Stephan_et_al_CogSci_2021.pdf}{}}
\item
  Invited Talk {[}virtual{]} at
  \href{https://www.uni-goettingen.de/en/colloquium/355791.html}{Becog
  Colloquium}: \textbf{Computational/ mathematical modeling in cognitive
  science} (April 2021, Göttingen)
\item
  Invited Talk {[}virtual{]} at the \href{https://cpilab.org/}{CPI Lab}
  Tübingen: \textbf{The interplay between covariation, temporal, and
  mechanism information in singular causation judgments} (Jan 2021,
  London)
\end{itemize}

\hypertarget{section-2}{%
\subsection{2020}\label{section-2}}

\begin{itemize}
\tightlist
\item
  Talk {[}virtual{]} at \href{http://cicl.stanford.edu/}{CIC-Lab}:
  \textbf{Interpolating Causal Mechanisms - The Paradox of Knowing More}
  (Apr 2020, Stanford University)
\end{itemize}

\hypertarget{section-3}{%
\subsection{2019}\label{section-3}}

\begin{itemize}
\item
  Talk at
  \href{https://www.psychologie.uni-heidelberg.de/ae/crisp/Home.html}{CRISP-Lab}:
  \textbf{What made this happen? A computational modeling approach to
  answering causal queries about singular cases} (Dez 2019, Heidelberg)
\item
  Talk at \href{https://www.psych.uni-goettingen.de/de/anap}{ANaP-Lab}:
  \textbf{Computational Modeling and Progress in Cognitive Science} (Oct
  2019, Göttingen)
\item
  Talk at \href{https://www.europeanspp.org/meetings.html}{ESPP 2019}:
  \textbf{The Role of Effect and Sample Size in Causal Induction} (Sep
  2019, Athens)
\item
  Poster at
  \href{https://www.ruhr-uni-bochum.de/philosophy/EuroCogSci2019/home}{Euro
  CogSci 2019}: \textbf{The Role of Effect and Sample Size in Causal
  Induction} (Sep 2019, Bochum)
  {\href{publications/Poster_EuroCogSci_2019.pdf}{}}
\end{itemize}

\begin{center}\rule{0.5\linewidth}{0.5pt}\end{center}

\hfill\break

\hypertarget{all-publications}{%
\section{All publications}\label{all-publications}}

\hypertarget{section-4}{%
\subsection{2023}\label{section-4}}

\begin{itemize}
\item
  \textbf{Stephan, S.} (2023). Revisiting the narrow latent scope bias
  in explanatory reasoning. \emph{Cognition}, 241,
  105630.{\href{publications/Stephan_2023_RevisitingTheNarrowLatentScopeBias.pdf}{}}
  {\href{https://simonstephan31.github.io/revisit_nlsbias/index.html}{}}
  {\href{https://osf.io/n94ed/}{}}{\href{https://psyarxiv.com/a8cve/}{}}
\item
  \textbf{Stephan, S.}, Engelmann, N., \& Waldmann, M. R. (2023). The
  perceived dilution of causal strength. \emph{Cognitive Psychology},
  140,
  101540.{\href{publications/Stephan_Engelmann_Waldmann_2023_strengthDilution.pdf}{}}
  {\href{https://simonstephan31.github.io/The-Dilution-of-Causal-Strength/index.html}{}}
  {\href{https://osf.io/nrv7p/}{}}
\end{itemize}

\hypertarget{section-5}{%
\subsection{2022}\label{section-5}}

\begin{itemize}
\item
  \textbf{Stephan, S.}, \& Waldmann, M. R. (2022). The interplay between
  covariation, temporal, and mechanism information in singular causation
  judgments. In A. Wiegmann, \& P. Willemsen (Eds.). \emph{Advances in
  Experimental Philosophy of Causation}. London, UK: Bloomsbury Press.
  {\href{https://psyarxiv.com/ucafn}{}}
\item
  \textbf{Stephan, S.}, \& Waldmann, M. R. (2022). The role of mechanism
  knowledge in singular causation judgments. \emph{Cognition}, 218,
  104924.
  {\href{publications/Stephan_Waldmann_2021_MechanismCognit.pdf}{}}
  {\href{https://osf.io/325pr/}{}}
  {\href{https://github.com/SimonStephan31/The-Role-of-Mechanism-Information-in-Singular-Causation}{}}
\end{itemize}

\hypertarget{section-6}{%
\subsection{2021}\label{section-6}}

\begin{itemize}
\item
  Skovgaard-Olsen, N., \textbf{Stephan, S.}, \& Waldmann, M. R. (2021).
  Conditionals and the hierarchy of causal queries. \emph{Journal of
  Experimental Psychology: General}, 150, 2472--2505.
  {\href{https://philarchive.org/archive/SKOCAT-2}{}}
  {\href{https://osf.io/fa9rj/}{}}
\item
  Gerstenberg, T., \& \textbf{Stephan, S.} (2021). A counterfactual
  simulation model of causation by omission. \emph{Cognition}, 216,
  104842.
  {\href{https://cicl.stanford.edu/papers/gerstenberg2021omission.pdf}{}}
  {\href{https://psyarxiv.com/wmh4c/}{}}
  {\href{https://osf.io/jubdz/}{}}
  {\href{https://github.com/cicl-stanford/omission}{}}
  {\href{https://osf.io/fu9rq}{}}
\item
  \textbf{Stephan, S.}, Placì, Sarah \& Waldmann, M. R. (2021).
  Evaluating general versus singular causal prevention. In T. Fitch, C.
  Lamm, H. Leder, \& K. Tessmar (Eds.), \emph{Proceedings of the 43rd
  Annual Conference of the Cognitive Science Society.} (pp.~1402--1408).
  Austin, TX: Cognitive Science Society.
  {\href{https://escholarship.org/uc/item/3z5870v4\#main}{}}
  {\href{https://simonstephan31.github.io/singular_prevention_proceedings/}{}}
  {\href{publications/Poster_Stephan_et_al_CogSci_2021.pdf}{}}
\item
  \textbf{Stephan, S.}, Tentori, K., Pighin, S., \& Waldmann, M. R.
  (2021). Interpolating causal mechanisms: The paradox of knowing more.
  \emph{Journal of Experimental Psychology: General}, 150(8), 1500-1527.
  {\href{https://doi.apa.org/fulltext/2021-13036-001.pdf}{}}
  {\href{https://osf.io/aqzps/}{}}
\end{itemize}

\hypertarget{section-7}{%
\subsection{2020}\label{section-7}}

\begin{itemize}
\item
  \textbf{Stephan, S.}, \& Waldmann, M. R. (2020). Causal scope and
  causal strength: The number of potential effects of a cause influences
  causal strength estimates. In S. Denison., M. Mack, Y. Xu, \& B.C.
  Armstrong (Eds.), \emph{Proceedings of the 42th Annual Conference of
  the Cognitive Science Society} (pp.~3426 - 3432). Austin, TX:
  Cognitive Science Society.
  {\href{publications/Stephan-Waldmann-2020-_CausalScopeAndStrength_Proceedings.pdf}{}}
  {\href{https://osf.io/mjswc/}{}}
\item
  \textbf{Stephan, S.}, \& Waldmann, M. R. (2020). On causal claims,
  contingencies, and inference: How causal terminology affects what we
  think about the strength of causal links. In S. Denison., M. Mack, Y.
  Xu, \& B.C. Armstrong (Eds.), \emph{Proceedings of the 42th Annual
  Conference of the Cognitive Science Society} (pp.~3419 - 3425).
  Austin, TX: Cognitive Science Society.
  {\href{publications/Stephan-Waldmann-2020-_CausalClaimsAndInference_Proceedings.pdf}{}}
  {\href{https://osf.io/5ngpd/}{}}
\item
  \textbf{Stephan, S.}, Mayrhofer, R., \& Waldmann, M. R. (2020). Time
  and singular causation - a computational model. \emph{Cognitive
  Science}, 44, e12871.
  {\href{https://onlinelibrary.wiley.com/doi/epdf/10.1111/cogs.12871}{}}
  {\href{https://osf.io/5yvs4/}{}} {\href{https://osf.io/n93bu}{}}
\end{itemize}

\hypertarget{section-8}{%
\subsection{2019}\label{section-8}}

\begin{itemize}
\tightlist
\item
  \textbf{Stephan, S.} (2019). Answering causal queries about singular
  cases - an evaluation of a new computational model. (Dissertation)
  {\href{publications/dissertation_simon_stephan.pdf}{}}
\end{itemize}

\hypertarget{section-9}{%
\subsection{2018}\label{section-9}}

\begin{itemize}
\item
  \textbf{Stephan, S.}, Mayrhofer, R., \& Waldmann, M. R. (2018).
  Assessing singular causation: The role of causal latencies. In T.T.
  Rogers, M. Rau, X. Zhu, \& C. W. Kalish (Eds.), \emph{Proceedings of
  the 40th Annual Conference of the Cognitive Science Society} (pp.~1080
  - 1085). Austin, TX: Cognitive Science Society.
  {\href{publications/Stephan-Mayrhofer-Waldmann-2018-_SingularCausationLatency.pdf}{}}
\item
  \textbf{Stephan, S.}, \& Waldmann, M. R. (2018). Preemption in
  singular causation judgments: A computational model. \emph{Topics in
  Cognitive Science}, 10, 242--257.
  {\href{https://onlinelibrary.wiley.com/doi/epdf/10.1111/tops.12309}{}}
\end{itemize}

\hypertarget{section-10}{%
\subsection{2017}\label{section-10}}

\begin{itemize}
\item
  \textbf{Stephan, S.}, \& Waldmann, M. R. (2017). Preemption in
  singular causation judgments: A computational model. In G. Gunzelmann,
  A. Howes, T. Tenbrink, \& E. Davelaar (Eds.), \emph{Proceedings of the
  39th Annual Conference of the Cognitive Science Society}
  (pp.~1126-1131). Austin, TX: Cognitive Science Society.
  (\textbf{Computational Modeling: Higher Level Cognition Award of the
  Cognitive Science Society}).
  {\includegraphics{images/home/Lit_award_icon.png}}
  {\href{publications/Stephan-Waldmann-2017-_Preemption\%20in\%20singular\%20causation\%20judgments-A\%20computational\%20model.pdf}{}}
\item
  \textbf{Stephan, S.}, Willemsen, P. \& Gerstenberg, T. (2017). Marbles
  in inaction: Counterfactual simulation and causation by omission. In
  G. Gunzelmann, A. Howes, T. Tenbrink, \& E. Davelaar (Eds.),
  \emph{Proceedings of the 39th Annual Conference of the Cognitive
  Science Society} (pp.~1132-1137). Austin, TX: Cognitive Science
  Society.
  {\href{publications/Stephan-Willemsen-Gerstenberg-2017-_omissions.pdf}{}}
\end{itemize}

\hypertarget{section-11}{%
\subsection{2016}\label{section-11}}

\begin{itemize}
\item
  Nagel, J., \& \textbf{Stephan, S.} (2016). Explanations in causal
  chains: Selecting distal causes requires exportable mechanisms. In A.
  Papafragou, D. Grodner, D. Mirman, \& J.C. Trueswell (Eds.),
  \emph{Proceedings of the 38th Annual Conference of the Cognitive
  Science Society} (pp.~806-811). Austin, TX: Cognitive Science Society.
  {\href{publications/Nagel-Stephan-2016-_Explanations\%20in\%20causal\%20chains_selecting\%20distal\%20causes.pdf}{}}
  {\href{https://www.psych.uni-goettingen.de/de/cognition/publikationen-dateien-stephan/2016_Poster_CogSci_Stephan.pdf}{}}
\item
  \textbf{Stephan, S.}, \& Waldmann, M. R. (2016). Answering causal
  queries about singular cases. In A. Papafragou, D. Grodner, D. Mirman,
  \& J.C. Trueswell (Eds.), \emph{Proceedings of the 38th Annual
  Conference of the Cognitive Science Society} (pp.~2795-2801). Austin,
  TX: Cognitive Science Society.
  {\href{publications/Stephan-Waldmann-2016_Answering\%20causal\%20queries\%20about\%20singular\%20cases.pdf}{}}
\end{itemize}

\hypertarget{section-12}{%
\subsection{2015}\label{section-12}}

\begin{itemize}
\tightlist
\item
  Nagel, J., \& \textbf{Stephan, S.}. (2015). Mediators or alternative
  explanations: Transitivity in human-mediated causal chains. In D. C.
  Noelle, R. Dale, A. S. Warlaumont, J. Yoshimi, T. Matlock, C. D.
  Jennings, \& P. P. Maglio (Eds.), \emph{Proceedings of the 37th Annual
  Meeting of the Cognitive Science Society} (pp.~1691-1696). Austin, TX:
  Cognitive Science Society.
  {\href{publications/Nagel-Stephan-2015_mediators.pdf}{}}
  {\href{https://www.psych.uni-goettingen.de/de/cognition/publikationen-dateien-stephan/2015_Poster_CogSci_Stephan.pdf}{}}
\end{itemize}

Back to Top ↑

\end{document}
